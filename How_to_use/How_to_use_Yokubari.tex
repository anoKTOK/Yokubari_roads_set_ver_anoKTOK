\documentclass{jbook}
\usepackage[dvipdfmx]{graphicx}
\usepackage[dvipdfmx]{color}
%\usepackage {okumacro}

%\usepackage{amsmath} % さまざまな数式を用いるための設定
%\usepackage{amsthm}
%\usepackage{mathrsfs}
%\usepackage[dvipdfmx]{pict2e}
\usepackage[dvipdfmx]{hyperref}
\usepackage{pxjahyper}		%日本語設定
\usepackage{multicol}
%\hypersetup{bookmarks=true}
%\hypersetup{bookmarksopen=true}
%\usepackage{emath}
%\usepackage{amsfonts}%白抜きbold作成
%\usepackage{url}%urlの記述
%\usepackage{makeidx} % 索引作成
%\usepackage{lscape} %landscaoeのときにpdfを反転する.
%\usepackage{longtable}	%tableに改頁

%\usepackage{tikz}%tikzを発動!
%\usepackage{tikz-cd}
%\usetikzlibrary{positioning}%
%\usetikzlibrary{matrix}
%\usepackage[abbrev]{amsrefs}


% ######## measure #########
% # mm = 1mm = 2.85pt      #
% # cm = 10mm = 28.5pt     #
% # in = 25.4mm = 72.27pt  #
% # pt = 0.35mm = 1pt      #
% # em = width of [M]      #
% # ex = height of [x]     #
% # zw = width of [Kanji]  #
% # zh = height of [Kanji] #
% ##########################
% ##################### Portrait Setting #########################
% # TOP = 1inch + \voffset + \topmargin + \headheight + \headsep #
% #     = 1inch + 0pt + 4pt + 20pt + 18pt (default)              #
% # BOTTOM = \paperheight - TOP -\textheight                     #
% ################################################################
\setlength{\textheight}{\paperheight}   % 紙面縦幅を本文領域にする(BOTTOM=-TOP)
\setlength{\topmargin}{4.6truemm}       % 上の余白を30mm(=1inch+4.6mm)に
\addtolength{\topmargin}{-\headheight}  %
\addtolength{\topmargin}{-\headsep}     % ヘッダの分だけ本文領域を移動させる
\addtolength{\textheight}{-60truemm}    % 下の余白も30mm(BOTTOM=-TOPだから+TOP+30mm)
% #################### Landscape Setting #######################
% # LEFT = 1inch + \hoffset + \oddsidemargin (\evensidemargin) #
% #      = 1inch + 0pt + 0pt                                   #
% # RIGHT = \paperwidth - LEFT - \textwidth                    #
% ##############################################################
\setlength{\textwidth}{\paperwidth}     % 紙面横幅を本文領域にする(RIGHT=-LEFT)
\setlength{\oddsidemargin}{-0.4truemm}  % 左の余白を25mm(=1inch-0.4mm)に
\setlength{\evensidemargin}{-0.4truemm} %
\addtolength{\textwidth}{-50truemm}     % 右の余白も25mm(RIGHT=-LEFT)

\setlength{\columnseprule}{.4pt}

%%%%%%%%%%%%%%%%%%%%%%%%%%%%%%%%%%%%%%%%%%%%%%%%%%%%%%%%%%%%%
%↑priamble
%
%%%%%%%%%%%%%%%%%%%%%%%%%%%%%%%%%%%%%%%%%%%%%%%%%%%%%%%%%%%%%
\begin{document}

\thispagestyle{empty}

\begin{flushleft}

  { \fontsize{24pt}{0pt}\selectfont
    Yokubari道路セットの取り扱い説明書
  }
	\\
\vspace{3pt}
	{ \fontsize{12pt}{0pt}\selectfont
		How to use Yokubari roads set
	}
	\\
\vspace{15pt}
	{ \fontsize{18pt}{0pt}\selectfont
		著者: あのKTOK
	}
	\\
\vspace{2pt}
	{ \fontsize{12pt}{0pt}\selectfont
		Author: ano KTOK
	}
	\\
\vspace{15pt}
	{ \fontsize{12pt}{0pt}\selectfont
		Date: 2020/Feb./14
	}
\end{flushleft}

\vspace{15pt}
\begin{flushleft}
  \includegraphics[width = 15cm]{picture/20210207app1.png}
\end{flushleft}


\newpage

\renewcommand{\contentsname}{目次 / Index}
\tableofcontents

\newpage

\chapter{日本語版}

\section{はじめに}
本プロジェクトは都市開発・輸送シミュレーションゲームであるSimutransにアドオンを追加することで、道路をメインにかつて無いリアリティーをもってインフラを再現することを目的とする。

これらアドオンは全て統一規格のもとに開発され、相互に組み合わせることで拡張性と景観表現の自由度を向上させる。
この一種のモジュール構造ともいえるような実装方法をとった結果、景観表現力と引き換えに極端にユーザビリティを犠牲にせざるを得なかったため、万人に受け入れられるものではない。
しかし、それでもなお、リアリティーを追求してやまないインフラに拘りを持つ同志に利用していただければこれ幸いである。

\subsection*{前提知識}
 Simutransの基本的な操作方法(道路の引き方、消し方)、アドオンの導入の仕方は既知とします。

\subsection*{Pakファイル}
 128standard版は\href{https://github.com/anoKTOK/Yokubari_roads_set_ver_anoKTOK/tree/main/128standard/pakset}{128standard/pakset}に置いています。makeobjは60-0120.2で開けるを使用しています。

 128extended版は\href{https://github.com/anoKTOK/Yokubari_roads_set_ver_anoKTOK/tree/main/128extended/pakset}{128extended/pakset}に置いています。makeobjは2019年6月に更新されたものです。このアドオン群は、pak128.britain-ex-nightly、pak128{\_}Sweden{\_}Exなど、Pakサイズが128であれば動作することを確認しています。

\subsection*{ja.tab, en.tab}
128standard版は\href{https://github.com/anoKTOK/Yokubari_roads_set_ver_anoKTOK/tree/main/128standard/pakset/text}{128standard/pakset/text}に置いています。

128extended版は\href{https://github.com/anoKTOK/Yokubari_roads_set_ver_anoKTOK/tree/main/128extended/pakset/text}{128extended/pakset/text}に置いています。


\subsection*{ライセンス}
特記なき限り本リポジトリ内のpak及びdat, pngファイルは
"クリエイティブ・コモンズ 表示 - 非営利 - 継承 4.0 国際 ライセンス"
によって許諾されています。ライセンスの内容を知りたい方はコモンズ証をご確認ください。

コモンズ証:\href{http://creativecommons.org/licenses/by-nc-sa/4.0/deed.ja}{http://creativecommons.org/licenses/by-nc-sa/4.0/deed.ja}

リーガルコード:\href{http://creativecommons.org/licenses/by-nc-sa/4.0/legalcode}{http://creativecommons.org/licenses/by-nc-sa/4.0/legalcode}

\subsection*{注意事項}

\begin{description}
  \item[(1)]
    本paksetの中には製作者がKTOK以外に関係する方が居られます。引用する際は各パックセットの著者・編集者(copyright)をご確認の上、参照下さい。
  \item[(2)]
    pak更新にて、前のデータが読み込めないということはないように配慮します。しかし、パックセットによっては、描画の変更があります。ご了承ください。
\end{description}

\subsection*{連絡}
万が一、製品に不具合がございましたら、ご連絡下さい。 ただし、本説明書を読んでからでお願いいたします。 場合によっては連絡できないことがありますのでご了承下さい。

\href{https://twitter.com/ano_KTOK_}{@ano\_KTOK\_}

\newpage

\section{アドオン一覧}



\subsubsection*{道路ツール / Road Tools}

\begin{flushleft}
  \includegraphics{picture/menu-1-1.png}
\end{flushleft}
\begin{tabular}{rll}
  番号 & 道路 & 詳細 \\
  & 一般道 & 地上 \\
  1 & 2車線 市道 & \\
  2 & 2車線 黄線  & \\
  3 & 2車線 破線 & \\
  & 高速道路 & 地上 \\
  4 & 1車線 & 奥 角が丸い \\
  5 & 1車線 & 手前 角が丸い \\
  6 & 1車線 & 奥 幅広路肩 角が丸い \\
  7 & 1車線 & 手前 幅広路肩 角が丸い \\
  8 & 1車線 & 奥 角が直線 and 分岐奥\\
  9 & 1車線 & 手前 角が直線 and 分岐手前\\
  10 & 1車線 & 奥 幅広路肩 角が直線  and 分岐奥\\
  11 & 1車線 & 手前 幅広路肩 角が直線 and 分岐手前\\
  12 & 2車線 黄線 & \\
  12 & 2車線 白線 & \\
  13 & 2車線 破線 & \\
  & 高速道路 & 高架 \\
  14 & 1車線 & 奥 角が丸い \\
  15 & 1車線 & 手前 角が丸い \\
  16 & 1車線 & 奥 幅広路肩 角が丸い \\
  17 & 1車線 & 手前 幅広路肩 角が丸い \\
  18 & 1車線 & 奥 角が直線 and 分岐奥\\
  19 & 1車線 & 手前 角が直線 and 分岐手前\\
  20 & 1車線 & 奥 幅広路肩 角が直線 and 分岐奥 \\
  21 & 1車線 & 手前 幅広路肩 角が直線 and 分岐手前\\
  22 & 2車線 黄線 & \\
  23 & 2車線 白線 & \\
  24 & 2車線 破線 & \\
\end{tabular}

\vspace{5pt}
ライム色(明るい黄緑色)の//が付いているものは角が直線になってます。

\newpage

\subsubsection*{市電/軽便鉄道ツール / Trams/light rail Tools}
\begin{flushleft}
  \includegraphics{picture/menu-2-1.png}
\end{flushleft}
\begin{tabular}{rll}
  番号 & 道路 & 詳細 \\
  1 & 1車線 & 奥 角が丸い \\
  2 & 1車線 & 奥 角が直線 and 行き止まりが2車線から1車線 \\
  3 & 1車線 & 手前 角が丸い \\
  4 & 1車線 & 手前 角が直線 and 行き止まりが2車線から1車線 \\
  5 & 2車線 & 角が丸い \\
  6 & 2車線 & 角が直線 and 2車線から1+1車線 分岐 合流 \\
  7 & 2車線 & 側壁 関空連絡橋仕様 奥 角が丸い \\
  8 & 2車線 & 側壁 関空連絡橋仕様 手前 角が丸い ※Extendedだとfrontimageが機能しません\\
  9 & 2車線 & 曲面防音壁 奥\\
  10 & 2車線 & 曲面防音壁 手前 ※Extendedだとfrontimageが機能しません\\
  11 & 2車線 + 登板車線 & 本線2車線 - 補助1車線 合流\\
  12 & 2車線 + 登板車線 & 本線2車線 - 補助1車線 分岐\\
  13 & 3車線化 & 本線2車線 - 補助2車線 合流\\
  14 & 3車線化 & 本線2車線 - 補助2車線 分岐\\
  15 & & 本線2車線 - 補助1車線 右側合流 and 2車線から1車線\\
  16 & & 本線2車線 - 補助1車線 右側分岐 and 2車線から1車線\\
  17 & & 本線3車線 - 補助1車線 合流\\
  18 & & 本線3車線 - 補助1車線 分岐\\
\end{tabular}

\vspace{5pt}
朱色の//が付いているものは角が直線になってます。//が付いている者同士と相性がいいです。

7~10はExtendedだと、今後なくなる可能性があります。


\newpage

\section{組み合わせ}
\subsection*{一覧}


  \includegraphics[width = 135mm]{picture/20210214-road-1-9.png}

  \ref{sub:IC_at_center}へ\\

  \includegraphics[width = 75mm]{picture/20210214-road-2-8.png}

  \ref{sub:1to1and1}へ\\

  \includegraphics[width = 75mm]{picture/20210214-road-3-6.png}

  \ref{sub:2lane_to_1lane+1lane}へ\\

  \includegraphics[width = 75mm]{picture/20210214-road-4-6.png}

  \ref{sub:1lane_to_2lane}へ\\

  \includegraphics[width = 75mm]{picture/20210214-road-4-7.png}

  \ref{sub:1lane_to_2lane}へ\\

\includegraphics[width = 75mm]{picture/20210214-road-5-5.png}

  \ref{sub:1lane_to_2lane2}へ\\

\newpage

\subsection{道路の中央でIC}
\label{sub:IC_at_center}

  \includegraphics[width = 135mm]{picture/20210214-road-1-9.png}

  \vspace{10pt}

  (1)

  \includegraphics[width = 135mm]{picture/20210214-road-1-1.png}


  まず、道路を引きます。
  2車線道路を2つ5マス以上引きます。
  \\

  (2)

  \includegraphics[width = 135mm]{picture/20210214-road-1-2.png}


  合流したいところで1車線道路でつなげて、伸ばしたい方向に1マス引きます。
  \\

\newpage
  (3)

  \includegraphics[width = 135mm]{picture/20210214-road-1-3.png}


  さっき、伸ばしたところから、2車線道路を上書きして引きます。
  \\


  (3)

  \includegraphics[width = 135mm]{picture/20210214-road-1-4.png}

  市電/軽便鉄道ツールを開きます。
  2車線の側壁(普通のもの)を合流部分から1マス以内のところを引きます。
  \\

  (4)

  \includegraphics[width = 135mm]{picture/20210214-road-1-5.png}

  伸ばしていない方向に、「本線2車線 - 補助1車線」アドオンを上書きします。
  \\

\newpage
  (5)

  \includegraphics[width = 135mm]{picture/20210214-road-1-6.png}

  反対側も同じく上書きします。
  \\

  (6)

  \includegraphics[width = 135mm]{picture/20210214-road-1-7.png}

  「本線2車線 - 補助1車線 右側〇〇」を西から真ん中2マスに上書きします。
  \\

  (7)

  \includegraphics[width = 135mm]{picture/20210214-road-1-8.png}

  反対側も同じです。
  \\

\newpage
  (8)

  \includegraphics[width = 135mm]{picture/20210214-road-1-9.png}

  最後に、2車線の[//]が付いている側壁アドオンで、中央から伸ばしたい方向に上書きすると完成です。

\newpage

\subsection{1車線から1車線+1車線の分岐}
\label{sub:1to1and1}

\includegraphics[width = 75mm]{picture/20210214-road-2-8.png}
\begin{multicols}{2}


  (1)

\includegraphics[width = 75mm]{picture/20210214-road-2-1.png}

道路ツールを開きます。
1車線(狭い方)奥を5マスくらい引きます。\\


(2)

\includegraphics[width = 75mm]{picture/20210214-road-2-2.png}

「1車線 手前」の「//」が付いているアドオンでで、分岐させたいところから、分岐させる方向に$+1$マス上書きします。
分岐させたい方向が違う場合は、「1車線 奥」の「//」が付いているアドオンで上書きします。\\

「1車線」の「//」が付いているアドオンのT字路についてです。手前側に分岐させたものは「1車線 手前」、奥側に分岐させたものは「1車線 奥」で設定しています。\\

(3)


\includegraphics[width = 75mm]{picture/20210214-road-2-3.png}

「//」が付いているアドオンを使って分岐させた方向を伸ばします。

「//」が付いているアドオンは角の斜めでは直線になります。\\


(4)

\includegraphics[width = 75mm]{picture/20210214-road-2-4.png}

市電/軽便鉄道ツールを開きます。
奥側の道路は何も付いていない「1車線 奥」の側壁を引きます。\\

\newpage

(5)

\includegraphics[width = 75mm]{picture/20210214-road-2-5.png}

分岐する、1マス手前のところを、「1車線 奥」の「//」が付いている側壁アドオンで上書きします。

ここで、「1車線」の「//」が付いている側壁アドオンの行き止まりは、2車線から1車線で登録しています。
1車線になるが奥なら「1車線 奥」、手前なら「1車線 手前」で設定しています。\\

もし、南側に側壁を延伸する場合は、市電の鉄道をつなげないでください。画像が変わります。

(6)

\includegraphics[width = 75mm]{picture/20210214-road-2-6.png}

「1車線 手前」の「//」が付いている側壁アドオンでで、分岐させたいところから、分岐させる方向に$+1$マス上書きします。
分岐させたい方向が違う場合は、「1車線 奥」の「//」が付いている側壁アドオンで上書きします。

「1車線」の「//」が付いている側壁アドオンのT字路についてです。手前側に分岐させたものは「1車線 手前」、奥側に分岐させたものは「1車線 奥」で設定しています。\\

%\includegraphics[width = 75mm]{picture/20210214-road-2-7.png}

(7)

\includegraphics[width = 75mm]{picture/20210214-road-2-8.png}

伸ばしたところを調節します。\\

\end{multicols}


\newpage

\subsection{2車線から1車線と1車線へ分岐}
\label{sub:2lane_to_1lane+1lane}

\includegraphics[width = 75mm]{picture/20210214-road-3-6.png}
\vspace{10pt}

\begin{multicols}{2}

  (1)

\includegraphics[width = 75mm]{picture/20210214-road-3-1.png}

道路ツールを開きます。
1車線 奥 幅広路肩のアドオンを6マスくらい引きます。\\

(2)

\includegraphics[width = 75mm]{picture/20210214-road-3-2.png}

「1車線 手前 幅広路肩」の「//」が付いているアドオンでで、分岐させたいところから、分岐させる方向に$+1$マス上書きします。
分岐させたい方向が違う場合は、「1車線 奥 幅広路肩」の「//」が付いているアドオンで上書きします。

「1車線」の「//」が付いているアドオンのT字路についてです。手前側に分岐させたものは「1車線 手前 幅広路肩」、奥側に分岐させたものは「1車線 奥 幅広路肩」で設定しています。\\

(3)

\includegraphics[width = 75mm]{picture/20210214-road-3-3.png}

「//」が付いているアドオンを使って分岐させた方向を伸ばします。

「//」が付いているアドオンは角の斜めでは直線になります。\\


(4)

\includegraphics[width = 75mm]{picture/20210214-road-3-4.png}

2車線道路をデコります。\\

\newpage
(5)

\includegraphics[width = 75mm]{picture/20210214-road-3-5.png}

市電/軽便鉄道ツールを開きます。
2車線側壁の何も付いていないアドオンを道路全体に引きます。


(6)

\includegraphics[width = 75mm]{picture/20210214-road-3-6.png}

角の部分を2車線側壁の「//」が付いているアドオンで上書きします。

\end{multicols}

\newpage


\subsection{1車線から2車線へ}
\label{sub:1lane_to_2lane}

\begin{multicols}{2}
  \includegraphics[width = 75mm]{picture/20210214-road-4-6.png}

  \includegraphics[width = 75mm]{picture/20210214-road-4-7.png}
\end{multicols}

\vspace{10pt}



\begin{multicols}{2}

  (1)

\includegraphics[width = 75mm]{picture/20210214-road-4-1.png}

道路ツールを開きます。
1車線 奥 幅広路肩の道路を6マスくらい引きます。\\


(2)

\includegraphics[width = 75mm]{picture/20210214-road-4-2.png}

車線幅を変更する1マス後のところに2車線道路を上書きします。\\


(3)

\includegraphics[width = 75mm]{picture/20210214-road-4-3.png}

市電/軽便鉄道ツールを開きます。
車線幅を変更するマスに、「本線2車線 - 補助1車線 右側〇〇」アドオンを引きます。

SISのオリジナルでは側壁が付いていないですが、側壁を付けてみました。\\

\newpage
(4)

\includegraphics[width = 75mm]{picture/20210214-road-4-4.png}

車線幅を変更する1マス後のところに2車線の側壁アドオンを上書きします。\\


(5-1)

\includegraphics[width = 75mm]{picture/20210214-road-4-6.png}

車線幅を変更する1マス前から南に2車線側壁のアドオンを引きます。
車線幅を変更するマスとはくっつけない場合はこうなります。\\

(5-2)

\includegraphics[width = 75mm]{picture/20210214-road-4-7.png}

つなげるとこうなります。

\end{multicols}

\newpage

\subsection{1車線から2車線へ2}
\label{sub:1lane_to_2lane2}

\includegraphics[width = 75mm]{picture/20210214-road-5-5.png}

\vspace{10pt}

\begin{multicols}{2}

  (1)

\includegraphics[width = 75mm]{picture/20210214-road-5-1.png}

道路ツールを開きます。
1車線 奥の道路を6マスくらい引きます。\\

(2)

\includegraphics[width = 75mm]{picture/20210214-road-5-2.png}

車線幅を変更する1マス後のところに2車線道路を上書きします。\\

(3)

\includegraphics[width = 75mm]{picture/20210214-road-5-3.png}

市電/軽便鉄道ツールを開きます。
車線幅を変更するマスに、「1車線 奥」の「//」が付いている側壁アドオンを引きます。



(4)

\includegraphics[width = 75mm]{picture/20210214-road-5-4.png}

車線幅を変更する1マス後のところに2車線の側壁アドオンを上書きします。

\newpage

(5)

\includegraphics[width = 75mm]{picture/20210214-road-5-5.png}

車線幅を変更するマスより前から南へ1車線の側壁アドオンを引きます。
このとき、車線幅を変更するマスと接続しないように注意します。


\end{multicols}

\newpage

\section{参考資料}
\begin{itemize}
  \item Project SIS (Ebi/Shirakami)
  \item csv2dat (Phystam)
\end{itemize}

\section{編集後記}

扱いに慣れるための有効だと思われる練習方法として以下をあげます。
\begin{itemize}
  \item
  $8$マスの四角を引いて、各辺の中心から四角形の外側に向かって$1$マス引いて見比べる。
  \item
  手前か奥かわからなければ、とりあえず思う方を引いておいて、違ったら別の方で上書きすると覚える量が減る。
  \item
  引きまくって慣れる。そのために、引かざるを得ない環境に置く。
  例えば、高速道路縛りで高速道路やIC、JTCをたくさん引く必要にかられるようにするのはありかもしれません。
  \item
  モチベーションを保つ。
  おそらく、Yokubari道路アドオンを使いたい人は、リアリティーのある道路を引きたい、見たいがコアにあるとあります。
\end{itemize}
残念ながらいくら頑張っても慣れない場合は、現状としてスクリプトを書いてくれる人を待つしか無いようです。

\vspace{15pt}
このアドオン群ですら足りないと思った方々は、すばらしい。その熱意をアドオン制作に注ぎ込むことを強く推奨します。
そして、公開して頂くと喜びに満ち溢れることでしょう。

\newpage

\chapter{English version}

\section{Introduction}

The aim of this project is to add addons to Simutrans, an urban development and transportation simulation game, to recreate infrastructure with unprecedented realism, mainly roads.

All of these addons are developed under a unified standard and can be combined with each other to improve scalability and flexibility of landscape expression.
As a result of this kind of modular implementation, we had to sacrifice extreme usability in exchange for the ability to express the scenery, so it is not something that can be accepted by everyone.
Nevertheless, we hope if it could be used by comrades who are concerned about infrastructure and who are still in pursuit of reality.

\subsection*{Previous knowledge}

We assume that you can draw and erase roads about Simutrans, enough.

\subsection*{Pak file}

The pak files of pak128 standard is put in \href{https://github.com/anoKTOK/Yokubari_roads_set_ver_anoKTOK/tree/main/128standard/pakset}{128standard/pakset}.
The makeobj is version 60-0.

 The pak files of pak128 extended is put in \href{https://github.com/anoKTOK/Yokubari_roads_set_ver_anoKTOK/tree/main/128extended/pakset}{128extended/pakset}.
 The makeobj to construct these addons is updated June 2019.
 I confirmed that the addons work in arbitrary addons of pak128 (for example pak128.britain-ex-nightly, pak128{\_}Sweden{\_}Ex).

\subsection*{ja.tab, en.tab}
 The en.tab of pak128 standard is put in \href{https://github.com/anoKTOK/Yokubari_roads_set_ver_anoKTOK/tree/main/128standard/pakset/text}{128standard/pakset/text}.

  The en.tab of pak128 extended is put in \href{https://github.com/anoKTOK/Yokubari_roads_set_ver_anoKTOK/tree/main/128extended/pakset/text}{128extended/pakset/text}.


\subsection*{License}
Attribution-NonCommercial-ShareAlike 4.0 International

Legal code:\href{http://creativecommons.org/licenses/by-nc-sa/4.0/legalcode}{http://creativecommons.org/licenses/by-nc-sa/4.0/legalcode}


\subsection*{Caution}

\begin{description}
  \item[(1)]
    If you remodel or re-distribute this addons, please confirm and cite the copyright.
  \item[(2)]
    By update, the pictures may change suddenly. Please keep this in mind.
\end{description}

\subsection*{Contact}

If you have some questions and
you read this article,
please contact us.

\href{https://twitter.com/ano_KTOK_}{@ano\_KTOK\_}

\newpage

\end{document}
